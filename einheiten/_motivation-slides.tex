
\section{Motivation}

\begin{frame}{TEI, now what?}
\metroset{block=fill}
    Why are we doing this workshop? The motivation from our abstract:
    \begin{itemize}
        \item \punkti the Text Encoding Initiative (TEI) for XML has become the gold standard for scholarly editions of texts.
        \item But what happens after an edition is encoded in TEI? 
        \item While it is an \textbf{ideal format for archiving digital data}, it is \alert{less than ideal for viewing and interacting with the edited text.}
    \end{itemize}
    
    \begin{block}{Goals for the next session}
    \begin{enumerate}
        \item understand the terms from the abstract:
        \begin{itemize}
            \item[\textcolor{alert}{\faClose}] digital edition
            \item[\textcolor{alert}{\faClose}] single source principle
            \item[\textcolor{alert}{\faClose}] \sout{creating different representations from our data} $\to$ (why) do we even need to create new presentations from our data? Aren't there tools for that?
        \end{itemize}
    \end{enumerate}
    \end{block}
\end{frame}

%--------------------------------------------
\begin{frame}{TEI, now what?}
\metroset{block=fill}
    Why are we doing this workshop? The motivation from our abstract:
    \begin{itemize}
        \item \punkti the Text Encoding Initiative (TEI) for XML has become the gold standard for scholarly editions of texts.
        \item \dots
    \end{itemize}
    
    \begin{block}{Goals for the next session}
    \begin{enumerate}
        \item wait, what was\dots
        \begin{itemize}
            \item[\textcolor{alert}{\faClose}] XML?
            \item[\textcolor{alert}{\faClose}] TEI?
            \item[\textcolor{alert}{\faClose}] How do I use the TEI for digital editing?
        \end{itemize}
    \end{enumerate}

    \end{block}
\end{frame}