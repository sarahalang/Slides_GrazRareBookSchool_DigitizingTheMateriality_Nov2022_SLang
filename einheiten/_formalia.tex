

\section{Syllabus}
%------------------------------------------------------------------------------
\begin{frame}{Content}
\begin{block}{Theoretical and practical introduction to digital scholarly editing:}

\begin{enumerate}\small
    \item what is a digital scholarly edition?
    \item who do I encode an edition in XML/TEI?
    \item what paradigms exist in digital scholarly editing?
\end{enumerate}
\end{block}

\footnotesize
Preparing historical documents for research is a core task in many humanities disciplines. The result of this work, the `scholarly edition', has seen fundamental changes due the use of digital technologies: Not only did a new, digital publishing medium take the place of paper, but new forms of analysis, semantization, research and the production of multiple representation formats based on one edition (text) source have been developed (\emph{single source principle}). 
The course serves as an introduction into the scholarly debate on the issue. It covers both theoretical foundations as well as working on practical examples coming from the participants' scholarly domain of origin. Furthermore, it introduces fundamental technologies like XML/TEI or transcription using Transkribus.

\begin{block}{Main learning goal}
Participants are able to recognize basic approaches and issues in the context of digital edition and to apply them to their own scientific domain in practical work.
\end{block}

\end{frame}
 
\section{Preliminaries}
%------------------------------------------------------------------------------
\begin{frame}{How to get a positive grade on this class}
\subsection{Grading}

  \begin{columns}[T,onlytextwidth]
    \column{0.48\textwidth}
      \begin{exampleblock}{Final Submission (60\%)}
\begin{itemize}\footnotesize
\item \textbf{Infomod:} Text, ER model, SQL database
\item \textbf{DigEd:} Small digital edition or review of an existing edition.
\item You can start in the last month of the semester and ask questions.
\item You can collaborate but no plagiarism (Uni Graz zero tolerance policy).
\end{itemize}
\end{exampleblock}

\begin{exampleblock}{Homework assignments (40\%)}\footnotesize
Communicated and to be completed within the week
\end{exampleblock}

    \column{0.48\textwidth}
      \begin{alertblock}{Other aspects}
\begin{enumerate}\scriptsize
    \item attendance in class (you can miss max. 3, to be communicated beforehand).
    \item Positive grade: at least 50\%  on all partial submissions.
    \item ``LVen mit immanentem Prüfungscharakter'' $\to$ once you accept the first task you get a grade (i.e. first homework this week)
    \item If you get a negative grade, the whole class needs to be retaken.
\end{enumerate}

\begin{quote}\scriptsize
    Nichterbringung weiterer Teilleistungen ohne wichtigen Grund ist Prüfungsabbruch (Negativbeurteilung). Abmeldung nach bereits übernommener Teilleistung führt zu negativer Beurteilung.
\end{quote}
\end{alertblock}

\end{columns}

\small 
see also: slides on grading \& further info materials on the final submission
    
\end{frame}


%------------------------------------------------------------------------------
\begin{frame}{Deadlines}

\begin{alertblock}{Hard deadlines}\small
\textbf{All deadlines are hard deadlines.}  You can get extensions for good reasons. 
\begin{itemize}
\item Good reasons for example: care responsibility, being ill, etc.
\item i.e. understandable reasons which are communicated asap
\end{itemize}

\end{alertblock}

\begin{alertblock}{If you miss a deadline\dots}
\begin{itemize}\small
\item If you didn’t communicate: negative grade.
\item Otherwise up for discussion according to the circumstances.
\end{itemize}
\end{alertblock}

\end{frame}
%------------------------------------------------------------------------------
\begin{frame}{Learning Goals}
\subsection{Learning Goals}
\begin{enumerate}
    \item Participants are able to recognize basic approaches and issues in the context of digital edition and to apply them to their own scientific domain in practical work.
    \begin{itemize}
        \item knowledge of theoretical basics
        \item being able to judge the quality of a digital scholarly edition
        \item being able to encode a DSE in XML/TEI
    \end{itemize}
    \item XML/TEI for digital scholarly editions
    \item some special cases such as encoding critical apparatus or zones
\end{enumerate}

\metroset{block=fill}
\begin{alertblock}{Final project}
\footnotesize
Small digital edition or review of an existing edition following the reviewing guidelines by the journal RIDE. To be submitted February 15th.

\end{alertblock}
\end{frame}


%------------------------------------------------------------------------------
\begin{frame}{Working with computers as a humanities person}

\begin{itemize}
    \item \href{https://static.uni-graz.at/fileadmin/gewi-zentren/Informationsmodellierung/PDF/U__bungsblatt-0.pdf}{Übungsblatt 0} is a prerequisite
    \item don't panic
\end{itemize}
    
\end{frame}

%------------------------------------------------------------------------------
\begin{frame}[standout]
    \alert{Present yourselves! } \\
    \normalsize
    Name, pronouns, domain of origin, interests, etc. \\
    Previous knowledge in Digital Humanities, scholarly editing, XML/TEI?
\end{frame}


%------------------------------------------------------------------------------
%\begin{frame}[standout]    Resources\end{frame}
\subsection{Resources}
%------------------------------------------------------------------------------
\begin{frame}[allowframebreaks]{References}
\begin{enumerate}\footnotesize
    \item Patrick \textbf{Sahle:} \emph{Digitale Editionsformen. Zum Umgang mit der Überlieferung unter den Bedingungen des Medienwandels, Schriften des Instituts für Dokumentologie und Editorik} (Norderstedt, 2013)
    \item Patrick \textbf{Sahle:} \emph{Digitale Editionstechniken}, in: Martin Gasteiner und Peter Haber (Hrsg.), \emph{Digitale Arbeitstechniken für die Geistes- und Kulturwissenschaften.} Wien: UTB, 2009.
    \item Patrick \textbf{Sahle:} \emph{Zwischen Mediengebundenheit und Transmedialisierung. Anmerkungen zum Verhältnis von Edition und Medien.} In: \emph{editio} 24 (2010), 23--36. DOI 10.1515/edit.2010.004
    \item Elena \textbf{Pierazzo:} \emph{What Future for Digital Scholarly Editions? From Haute Couture to Prêt-à-Porter}, \emph{International Journal of Digital Humanities} 1.2 (2019), 209--220. \protect\url{https://doi.org/10.1007/s42803-019-00019-3}
    \item Elena \textbf{Pierazzo:} \emph{Digital Scholarly Editing}, Farnham u.a. 2015.
    \item Matthew James \textbf{Driscoll} \& Elena \textbf{Pierazzo (eds.):} \emph{Digital Scholarly Editing. Theory, Practice and Future Perspectives} (Open Book Publishers, 2016). \protect\url{https://doi.org/10.11647/OBP.0095}
    \item Marilyn \textbf{Deegan} \& Kathryn \textbf{Sutherland (eds.):} Text Editing, print and the Digital World, Aldershof/UK 2009 (Digital Research in Arts and Humanities).
    \item Special Issue `Digital Scholarly Edition' of the \emph{International Journal of Digital Humanities} 1, Nr. 2 (2019); \href{https://link.springer.com/journal/42803/topicalCollection/AC_00bb5c504ba4a0bbaa9dc32f1881a986}{<Link>}.
    \item Mats \textbf{Dahlström:} How Reproductive is a Scholarly Edition? In: \emph{Literary and Linguistic Computing} 19/1 (2004), S. 17-33. 
    \item Mats \textbf{Dahlström:} ``Critical editing and critical digitization''. In: E. Thoutenhoofd, A. van der Weel \& W. Th. van Peursen (eds.), \emph{Text comparison and digital creativity: the production of presence and meaning in digital text scholarship.} Amsterdam: Brill, 2010, 79--97.
    \item Peter \textbf{Robinson:} ``Towards a Theory of Digital Editions'', in: \emph{Variants} 10 (2013).
    \item Susan \textbf{Schreibman:} \emph{Digital Scholarly Editing}, in: \emph{Literary Studies in the Digital Age: An evolving anthology}, Kenneth M. Price \& Ray Siemens (eds.), 2014.  \protect\url{http://dlsanthology.commons.mla.org/digital-scholarly-editing/}
    \item M. N. \textbf{Smith:} Electronic scholarly editing, in: A Companion to Digital Humanities, Susan Schreibman, Raymond George Siemans \& John Unsworth eds., Malden 2004, 306--322.
    \item More references: \protect\url{http://www.hgw-online.net/GHWBibliographie/systematik/Editionstechnik}
\end{enumerate}

\end{frame}

%------------------------------------------------------------------------------
\begin{frame}{Ressources}

\begin{enumerate}
    \item \textbf{OxygenXML:}
    \begin{itemize}
        \item \href{https://www.oxygenxml.com/}{oXygen XML editor}
        \item \href{https://www.oxygenxml.com/xml_editor/download_oxygenxml_editor.html}{Download oXygen XML}
        \item to use this, you will need a license key $\to$ Moodle
    \end{itemize}
    \item \textbf{XML/TEI}
    \begin{itemize}
        \item \href{https://tei-c.org/}{Text Encoding Initiative (TEI)}
        \item \href{https://de.wikipedia.org/wiki/Text_Encoding_Initiative}{TEI Wikipedia}
        \item German intro video using OxygenXML: \href{https://www.youtube.com/watch?v=fnVV9N4kkQ8}{TEI-Annotation in Oxygen XML: Von den Basics zur Automatisierung mit regulären Ausdrücken} $\to$ switch on auto-generated subtitles
        \item Blog post \href{https://latex-ninja.com/2022/02/02/a-shamelessly-short-intro-to-xml-for-dh-beginners-includes-tei/}{\emph{A shamelessly short intro to XML for DH beginners (includes TEI)}, \LaTeX{}-Ninja Blog} (02.02.2022).
        \item \href{https://www.w3schools.com/xml/}{W3Schools step-by-step XML tutorial}
    \end{itemize}
    \item \textbf{Critical Apparatus Toolbox:}
    \begin{itemize}
        \item \href{https://web.archive.org/web/20191211173459/http://teicat.huma-num.fr/}{Marjorie Burghart’s TEI Critical Apparatus Toolbox}
        \item \href{https://doi.org/10.4000/jtei.1520}{Marjorie Burghart, \emph{The TEI Critical Apparatus Toolbox: Empowering Textual Scholars through Display, Control, and Comparison Features}, in \emph{jTEI} 10: \protect{https://doi.org/10.4000/jtei.1520}}
    \end{itemize}
    
\end{enumerate}

\end{frame}

%------------------------------------------------------------------------------
\begin{frame}{First homework assignment}

\begin{itemize}
    \item Please read: \\ 
    Patrick \textbf{Sahle:} ``What is a scholarly digital edition (SDE)?'' In: \emph{Digital Scholarly Editing. Theory, Practice and Future Perspectives.} Ed. by Matthew Driscoll and Elena Pierazzo. Cambridge: Open Book Publishers, 2016, 19--39.  \protect\url{https://books.openedition.org/obp/3397?lang=en}
    \item Write a short reflection on the text in which you try to explain the difference between a digital and a non-digital edition.
    \item Make sure to keep your musings to maximum 1-2 pages of text! 
    \item The deadline is this Saturday night! (23:59)
    \item Careful -- by submitting this homework, you take on the first assignment and therefore, will receive a grade on this class even if you drop out!
\end{itemize}
    
\end{frame}

