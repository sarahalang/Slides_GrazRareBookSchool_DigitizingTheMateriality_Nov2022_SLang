



\section{Data modelling}


%------------------------------------------------------------------------------
\begin{frame}[allowframebreaks]{3 proporties of a model following Stachowiak}

\metroset{block=fill}
\begin{alertblock}{1)~ Mapping }
\begin{quote} \scriptsize
    „Modelle sind stets Modelle von etwas, nämlich Abbildungen, Repräsentationen natürlicher oder künstlicher Originale, die selbst wieder Modelle sein können.“ 
    %„Originale und Modelle werden hier ausschließlich als Attributklassen gedeutet, die oft die spezielle Gestalt attributiver Systeme erlangen.“ % 131
    „Der Abbildungsbegriff fällt mit dem Begriff der Zuordnung von Modell-Attributen zu Original-Attributen zusammen.“ \parencite[131--132]{stachowiak} % 132
    
    \alert{Models are always models of something, i.e. mappings from, representations of natural or artificial originals, that can be models themselves.}
\end{quote}
\end{alertblock}
\begin{alertblock}{2)~ Reduction}
\begin{quote} \scriptsize
    „Modelle erfassen im allgemeinen nicht alle Attribute des durch sie repräsentierten Originals, sondern nur solche, die den jeweiligen Modellerschaffern und/oder Modelbenutzern relevant scheinen.“ \parencite[132]{stachowiak}
    
    \alert{Models in general capture not all attributes of the original represented by them, but rather only those seeming relevant to their model creators and/ or model users.}
\end{quote}
\end{alertblock}
\framebreak 

\begin{alertblock}{3)~ Pragmatism}
\begin{quote} \scriptsize
    „Eine pragmatisch vollständige Bestimmung des Modellbegriffs hat nicht nur die Frage zu berücksichtigen, \emph{wovon} etwas Modell ist \lbrack{}Abbildungsmerkmal\rbrack{}, sondern auch, \emph{für wen, wann} und \emph{wozu} bezüglich seiner je spezifischen Funktionen es Modell ist.“ \parencite[132]{stachowiak} % 132
    
    \alert{ Models are not uniquely assigned to their originals per se. They fulfill their replacement function \\
    a) for particular – cognitive and/ or acting, model using subjects, \\
    b) within particular time intervals and \\
    c) restricted to particular mental or actual operations.}
\end{quote}
\end{alertblock}

\end{frame}

%------------------------------------------------------------------------------
\begin{frame}{Stachowiak's \emph{General Model Theory}}

\begin{block}{Stachowiak's notion of a model}
\begin{quote}
    Alle Erkenntnis ist Erkenntnis in Modellen oder durch Modelle und jede menschliche Weltbegegnung überhaupt bedarf des Mediums Modell.
\end{quote}
\alert{$\to$ All knowledge-making is knowledge-making in or through models and all human perception of the world needs models as a medium. }
\end{block}

\begin{block}{Data modelling}\small
Model = snippet of the real world but it only covers the attributes I chose to be relevant for the task at hand. 
Thus, the model and the aspect of the real world it models (its subject) diverge. 
\end{block}

\begin{block}{Digital, standardized \& formal models}\small 
Standardized models allow us to exchange and analyse data, search/query data. 
 $\to$ only \emph{formal models} can be processed digitally, i.e. every digital model is a formal model.
\end{block}

\alert{$\to$ Models are simplified representations of parts of the real world.}

\end{frame}

%------------------------------------------------------------------------------
\begin{frame}{Why should we care about modelling theory?}
    \begin{itemize}
        \item Modelling is a pivotal task in the Digital Humanities.
        \item When we create digital representations of material objects, those are models. 
        \item Models are by definition subjective, abstracted and not universal. 
        \item Their quality has to be judged in relation to their purpose (Stachowiak's 3rd criterion): $\to$
    \end{itemize}
\end{frame}


%------------------------------------------------------------------------------
    \begin{frame}{Different motivations for digital projects}
    \begin{block}{research-driven:}\small
        individualized for answering a research question, work-intensive \& relatively expensive.
    \end{block}

    \begin{block}{curation-driven}\small
        mass-digitization, cookie cutter approach which covers the most important elements for most use cases but can easily miss features relevant to subject-matter experts.
            \begin{itemize}\footnotesize
                \item despite the objects being digitized, users might have to go back to the material objects to fill in the blanks
                \item but objects become more discoverable \& (hopefully) somewhat comparable to larger corpora of similar objects
                \item superficial digitization can lead to the creation of misleading datasets, e.g. errors or bad tagging $\to$ \alert{reparative librarianship} is needed
                \item potential application for machine learning approaches
            \end{itemize}
    \end{block}

\end{frame}


%------------------------------------------------------------------------------
\begin{frame}{What are data?}
\metroset{block=fill}
\begin{block}{Definitions of `data'}
Plural of Latin \emph{datum} ('given'). 
\end{block}

\begin{itemize}\footnotesize 
    \item data aren't exactly a given but rather constructed or created.
    \item There has been a discussion in the DH whether we should call them \emph{capta}~\parencite{Drucker2011dataCapta}
    \item Most data (so-called `givens') are constructed by phenomenotechnical devices~\parencite{bachelard1968}, i.e. pereiving devices which translate the (often quite abtract) things they see into data.
    \item data has to be interpreted.
    \item Stachowiak's pragmatism criterion means we selectively capture data most important to us, not all possible aspects (!).
\end{itemize}

Data resulting from cataloging \& digitization contains interpretations \& is thus, 
\emph{per definitionem} always subjective \& incomplete (!).

\alert{data $\neq$ original!}

\end{frame}

