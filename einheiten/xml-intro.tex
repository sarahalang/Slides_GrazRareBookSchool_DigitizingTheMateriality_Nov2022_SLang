

\section{Annotating with XML markup}


%-----------------------------------------------------
\begin{frame}{XML: eXtensible Markup Language}
\begin{columns}
\column{0.35\textwidth}
\begin{itemize}\small 
    \item \href{https://www.w3schools.com/xml/default.asp}{W3Schools Tutorial} 
    \item {paradigm of the separation of form and content} 
    \item {XML is a metalanguage}
\end{itemize}
\bgupper{w3schools}{black}{.xml} \\

\begin{itemize}\scriptsize 
    \item {RSS}, SOAP, XAML 
    \item {MathML}, {GraphML}~ 
    \item {XHTML}~
    \item {RDF}~
    \item {KML}~ 
    \item {Scalable Vector Graphics (SVG)}
\end{itemize}

\column{0.65\textwidth}
\metroset{block=fill}
\begin{block}{}
\begin{quote}
    \textbf{Extensible Markup Language (XML)} is a \textbf{markup language} and file format for storing, transmitting, and reconstructing arbitrary data. It defines a \textbf{set of rules for encoding documents} in a format that is \textbf{both human-readable and machine-readable.}  (\href{https://en.wikipedia.org/wiki/XML}{Wikipedia})
\end{quote}
\end{block}
\end{columns}

\end{frame}


%----------------------------------

\begin{frame}[fragile]{XML rules}
\begin{columns}
\column{0.42\textwidth}
\small
XML can be checked for \textbf{validity} (validation if it complies with a standard) and \textbf{well-formedness} (following the rules of XML) $\to$ will only be parsed if well-formed. Thus: \alert{Heed thy error messages!}\smallskip

There are rules on how elements can be named (you can look them up if relevant or will get informed by an error message). 

\bigskip

%\includegraphics[width=0.1\textwidth]{doppelkeks.jpg}~\bg{alert}{white}{Doppelkeks} ~ ~\bg{alert}{white}{russische Puppe}~\includegraphics[width=0.15\textwidth]{matroschka.jpg}\vspace{1em}

\mycommand{<key>value</key>}{XML as a key value notation} 
\column{0.55\textwidth}\footnotesize
\metroset{block=fill}
\begin{block}{Rules}
\begin{itemize}
    \item Hierarchical nesting below the root
    \item exactly one root element, i.e. one out-most russion doll
    \item start and end tag 
    \item tag names are case-sensitive (!) 
    \item empty elements allowed (\& can be shortened) 
\end{itemize}
\end{block}
\bigskip 

\begin{block}{Minimal example}
\begin{xmlcode}
<?xml version="1.0" ?>
<root>
  <element attribute="value">
    content
  </element>
  <!-- comment -->
</root>
\end{xmlcode}
\end{block}
\end{columns}

\end{frame}


%----------------------------------


\begin{frame}[fragile,allowframebreaks]{XML rules}

\small

\bg{w3schools}{white}{Prolog}~ \\
\mycommand{<xml version="1.0" encoding="utf-8">}{XML declaration}
\mycommand{<?xsl-stylesheet type="text/xsl" href="my.xsl"?>}{processing instructions  (optional)}
\bigskip

you can include document models (optional) \\
DTD, XML Schema, RelaxNG, Schematron 
\bigskip

\bg{w3schools}{white}{entities}~ `protected' characters that have a meta meaning in XML like: \\
\mycommand{&lt;}{<}
\mycommand{&gt;}{>}
\mycommand{&amp;}{\&}


\end{frame}


%-----------------
\begin{frame}{XML family and vocabularies}
\begin{columns}
\column{0.45\textwidth}
\footnotesize
\bg{w3schools}{white}{XML}~structured description of data \\
\bg{w3schools}{white}{XPath}~navigating xml documents \\
\bg{w3schools}{white}{XML Schema}~strict data model \\
\bg{w3schools}{white}{XSL}~Extensible Stylesheet Language  \\
\bg{w3schools}{white}{XSLT}~XSL-Transformations, i.e. transforming XML documents  \\
\bg{w3schools}{white}{XSL-FO}~ formatted output (e.g. print) \\
\bg{w3schools}{white}{XQuery}~query language for XML databases \\
\bg{w3schools}{white}{and more}~
\column{0.45\textwidth}
\metroset{block=fill}
\begin{block}{}
\footnotesize
\begin{itemize}
    \item \textbf{(X)HTML} Hypertext Markup Language 
    \item \textbf{EAD} Encoded Archival Description 
    \item \textbf{TEI} Text Encoding Initiative 
    \item \textbf{CEI} Charters Encoding Initiative 
    \item \textbf{MEI} Music Encoding Initiative 
    \item \textbf{LIDO} Lightweight Information Describing Objects (describing museum or collection objects)
    \item \textbf{SVG} Scalable Vector Graphics 
    \item \textbf{KML} Keyhole Markup Language (geography)
    \item \textbf{MathML} 
    \item \textbf{CML} Chemical Markup Language, \dots
\end{itemize}
\end{block}
\end{columns}


\end{frame}


%------------------------------------------------------------------------------

\begin{frame}[standout]
  \alert{Practice!} \\
  \normalsize
  Open a new XML document in your editor (Oxygen). \\
  Create new elements and find 3 ways to break it so that you get an error. \\
  Then fix the error.

\end{frame}
