\documentclass[10pt]{beamer}
\input{CONFIG.tex}
\usepackage{multicol}
\usepackage{hyperref}
\usepackage[normalem]{ulem} % for strikethrough


\begin{document}
%------------------------------------------------------------------------------
% FRONT
%------------------------------------------------------------------------------
{
\usebackgroundtemplate{\includegraphics[width=\paperwidth]{img/kfu-zim-slides.png}}
\frame{\titlepage}
}
%\frame{\titlepage}
\setcounter{tocdepth}{1}
%------------------------------------------------------------------------------
\begin{frame}{Overview}%\scriptsize
  \setbeamertemplate{section in toc}[sections numbered]
  \begin{small}
  \begin{multicols}{2}\tableofcontents\end{multicols}
  % on toc formatting:
  % https://tex.stackexchange.com/questions/26929/two-column-beamer-toc-with-control-over-the-breaking-point
  \end{small}
\end{frame}
%------------------------------------------------------------------------------

\metroset{block=fill}
%

\section{Syllabus}
%------------------------------------------------------------------------------
\begin{frame}{Content}
\begin{block}{Theoretical and practical introduction to digital scholarly editing:}

\begin{enumerate}\small
    \item what is a digital scholarly edition?
    \item who do I encode an edition in XML/TEI?
    \item what paradigms exist in digital scholarly editing?
\end{enumerate}
\end{block}

\footnotesize
Preparing historical documents for research is a core task in many humanities disciplines. The result of this work, the `scholarly edition', has seen fundamental changes due the use of digital technologies: Not only did a new, digital publishing medium take the place of paper, but new forms of analysis, semantization, research and the production of multiple representation formats based on one edition (text) source have been developed (\emph{single source principle}). 
The course serves as an introduction into the scholarly debate on the issue. It covers both theoretical foundations as well as working on practical examples coming from the participants' scholarly domain of origin. Furthermore, it introduces fundamental technologies like XML/TEI or transcription using Transkribus.

\begin{block}{Main learning goal}
Participants are able to recognize basic approaches and issues in the context of digital edition and to apply them to their own scientific domain in practical work.
\end{block}

\end{frame}
 
\section{Preliminaries}
%------------------------------------------------------------------------------
\begin{frame}{How to get a positive grade on this class}
\subsection{Grading}

  \begin{columns}[T,onlytextwidth]
    \column{0.48\textwidth}
      \begin{exampleblock}{Final Submission (60\%)}
\begin{itemize}\footnotesize
\item \textbf{Infomod:} Text, ER model, SQL database
\item \textbf{DigEd:} Small digital edition or review of an existing edition.
\item You can start in the last month of the semester and ask questions.
\item You can collaborate but no plagiarism (Uni Graz zero tolerance policy).
\end{itemize}
\end{exampleblock}

\begin{exampleblock}{Homework assignments (40\%)}\footnotesize
Communicated and to be completed within the week
\end{exampleblock}

    \column{0.48\textwidth}
      \begin{alertblock}{Other aspects}
\begin{enumerate}\scriptsize
    \item attendance in class (you can miss max. 3, to be communicated beforehand).
    \item Positive grade: at least 50\%  on all partial submissions.
    \item ``LVen mit immanentem Prüfungscharakter'' $\to$ once you accept the first task you get a grade (i.e. first homework this week)
    \item If you get a negative grade, the whole class needs to be retaken.
\end{enumerate}

\begin{quote}\scriptsize
    Nichterbringung weiterer Teilleistungen ohne wichtigen Grund ist Prüfungsabbruch (Negativbeurteilung). Abmeldung nach bereits übernommener Teilleistung führt zu negativer Beurteilung.
\end{quote}
\end{alertblock}

\end{columns}

\small 
see also: slides on grading \& further info materials on the final submission
    
\end{frame}


%------------------------------------------------------------------------------
\begin{frame}{Deadlines}

\begin{alertblock}{Hard deadlines}\small
\textbf{All deadlines are hard deadlines.}  You can get extensions for good reasons. 
\begin{itemize}
\item Good reasons for example: care responsibility, being ill, etc.
\item i.e. understandable reasons which are communicated asap
\end{itemize}

\end{alertblock}

\begin{alertblock}{If you miss a deadline\dots}
\begin{itemize}\small
\item If you didn’t communicate: negative grade.
\item Otherwise up for discussion according to the circumstances.
\end{itemize}
\end{alertblock}

\end{frame}
%------------------------------------------------------------------------------
\begin{frame}{Learning Goals}
\subsection{Learning Goals}
\begin{enumerate}
    \item Participants are able to recognize basic approaches and issues in the context of digital edition and to apply them to their own scientific domain in practical work.
    \begin{itemize}
        \item knowledge of theoretical basics
        \item being able to judge the quality of a digital scholarly edition
        \item being able to encode a DSE in XML/TEI
    \end{itemize}
    \item XML/TEI for digital scholarly editions
    \item some special cases such as encoding critical apparatus or zones
\end{enumerate}

\metroset{block=fill}
\begin{alertblock}{Final project}
\footnotesize
Small digital edition or review of an existing edition following the reviewing guidelines by the journal RIDE. To be submitted February 15th.

\end{alertblock}
\end{frame}


%------------------------------------------------------------------------------
\begin{frame}{Working with computers as a humanities person}

\begin{itemize}
    \item \href{https://static.uni-graz.at/fileadmin/gewi-zentren/Informationsmodellierung/PDF/U__bungsblatt-0.pdf}{Übungsblatt 0} is a prerequisite
    \item don't panic
\end{itemize}
    
\end{frame}

%------------------------------------------------------------------------------
\begin{frame}[standout]
    \alert{Present yourselves! } \\
    \normalsize
    Name, pronouns, domain of origin, interests, etc. \\
    Previous knowledge in Digital Humanities, scholarly editing, XML/TEI?
\end{frame}


%------------------------------------------------------------------------------
%\begin{frame}[standout]    Resources\end{frame}
\subsection{Resources}
%------------------------------------------------------------------------------
\begin{frame}[allowframebreaks]{References}
\begin{enumerate}\footnotesize
    \item Patrick \textbf{Sahle:} \emph{Digitale Editionsformen. Zum Umgang mit der Überlieferung unter den Bedingungen des Medienwandels, Schriften des Instituts für Dokumentologie und Editorik} (Norderstedt, 2013)
    \item Patrick \textbf{Sahle:} \emph{Digitale Editionstechniken}, in: Martin Gasteiner und Peter Haber (Hrsg.), \emph{Digitale Arbeitstechniken für die Geistes- und Kulturwissenschaften.} Wien: UTB, 2009.
    \item Patrick \textbf{Sahle:} \emph{Zwischen Mediengebundenheit und Transmedialisierung. Anmerkungen zum Verhältnis von Edition und Medien.} In: \emph{editio} 24 (2010), 23--36. DOI 10.1515/edit.2010.004
    \item Elena \textbf{Pierazzo:} \emph{What Future for Digital Scholarly Editions? From Haute Couture to Prêt-à-Porter}, \emph{International Journal of Digital Humanities} 1.2 (2019), 209--220. \protect\url{https://doi.org/10.1007/s42803-019-00019-3}
    \item Elena \textbf{Pierazzo:} \emph{Digital Scholarly Editing}, Farnham u.a. 2015.
    \item Matthew James \textbf{Driscoll} \& Elena \textbf{Pierazzo (eds.):} \emph{Digital Scholarly Editing. Theory, Practice and Future Perspectives} (Open Book Publishers, 2016). \protect\url{https://doi.org/10.11647/OBP.0095}
    \item Marilyn \textbf{Deegan} \& Kathryn \textbf{Sutherland (eds.):} Text Editing, print and the Digital World, Aldershof/UK 2009 (Digital Research in Arts and Humanities).
    \item Special Issue `Digital Scholarly Edition' of the \emph{International Journal of Digital Humanities} 1, Nr. 2 (2019); \href{https://link.springer.com/journal/42803/topicalCollection/AC_00bb5c504ba4a0bbaa9dc32f1881a986}{<Link>}.
    \item Mats \textbf{Dahlström:} How Reproductive is a Scholarly Edition? In: \emph{Literary and Linguistic Computing} 19/1 (2004), S. 17-33. 
    \item Mats \textbf{Dahlström:} ``Critical editing and critical digitization''. In: E. Thoutenhoofd, A. van der Weel \& W. Th. van Peursen (eds.), \emph{Text comparison and digital creativity: the production of presence and meaning in digital text scholarship.} Amsterdam: Brill, 2010, 79--97.
    \item Peter \textbf{Robinson:} ``Towards a Theory of Digital Editions'', in: \emph{Variants} 10 (2013).
    \item Susan \textbf{Schreibman:} \emph{Digital Scholarly Editing}, in: \emph{Literary Studies in the Digital Age: An evolving anthology}, Kenneth M. Price \& Ray Siemens (eds.), 2014.  \protect\url{http://dlsanthology.commons.mla.org/digital-scholarly-editing/}
    \item M. N. \textbf{Smith:} Electronic scholarly editing, in: A Companion to Digital Humanities, Susan Schreibman, Raymond George Siemans \& John Unsworth eds., Malden 2004, 306--322.
    \item More references: \protect\url{http://www.hgw-online.net/GHWBibliographie/systematik/Editionstechnik}
\end{enumerate}

\end{frame}

%------------------------------------------------------------------------------
\begin{frame}{Ressources}

\begin{enumerate}
    \item \textbf{OxygenXML:}
    \begin{itemize}
        \item \href{https://www.oxygenxml.com/}{oXygen XML editor}
        \item \href{https://www.oxygenxml.com/xml_editor/download_oxygenxml_editor.html}{Download oXygen XML}
        \item to use this, you will need a license key $\to$ Moodle
    \end{itemize}
    \item \textbf{XML/TEI}
    \begin{itemize}
        \item \href{https://tei-c.org/}{Text Encoding Initiative (TEI)}
        \item \href{https://de.wikipedia.org/wiki/Text_Encoding_Initiative}{TEI Wikipedia}
        \item German intro video using OxygenXML: \href{https://www.youtube.com/watch?v=fnVV9N4kkQ8}{TEI-Annotation in Oxygen XML: Von den Basics zur Automatisierung mit regulären Ausdrücken} $\to$ switch on auto-generated subtitles
        \item Blog post \href{https://latex-ninja.com/2022/02/02/a-shamelessly-short-intro-to-xml-for-dh-beginners-includes-tei/}{\emph{A shamelessly short intro to XML for DH beginners (includes TEI)}, \LaTeX{}-Ninja Blog} (02.02.2022).
        \item \href{https://www.w3schools.com/xml/}{W3Schools step-by-step XML tutorial}
    \end{itemize}
    \item \textbf{Critical Apparatus Toolbox:}
    \begin{itemize}
        \item \href{https://web.archive.org/web/20191211173459/http://teicat.huma-num.fr/}{Marjorie Burghart’s TEI Critical Apparatus Toolbox}
        \item \href{https://doi.org/10.4000/jtei.1520}{Marjorie Burghart, \emph{The TEI Critical Apparatus Toolbox: Empowering Textual Scholars through Display, Control, and Comparison Features}, in \emph{jTEI} 10: \protect{https://doi.org/10.4000/jtei.1520}}
    \end{itemize}
    
\end{enumerate}

\end{frame}

%------------------------------------------------------------------------------
\begin{frame}{First homework assignment}

\begin{itemize}
    \item Please read: \\ 
    Patrick \textbf{Sahle:} ``What is a scholarly digital edition (SDE)?'' In: \emph{Digital Scholarly Editing. Theory, Practice and Future Perspectives.} Ed. by Matthew Driscoll and Elena Pierazzo. Cambridge: Open Book Publishers, 2016, 19--39.  \protect\url{https://books.openedition.org/obp/3397?lang=en}
    \item Write a short reflection on the text in which you try to explain the difference between a digital and a non-digital edition.
    \item Make sure to keep your musings to maximum 1-2 pages of text! 
    \item The deadline is this Saturday night! (23:59)
    \item Careful -- by submitting this homework, you take on the first assignment and therefore, will receive a grade on this class even if you drop out!
\end{itemize}
    
\end{frame}



%------------------------------------------------------------------------------
% MAIN
%------------------------------------------------------------------------------

%-----------------------------------------------------
\section{The TEI \& Digital Editing Session}

\begin{frame}{Goals}
\subsection{Goals}
\begin{enumerate}
    \item understand basic modelling theory for digitization 
    \item understand the differences between a digital scholarly edition \& other digital resources (such as archives or library catalog data)
    \item understand the role of XML/TEI in Digital Editing
    \item be able to use the Text Encoding Initiative standard to encode descriptions of pre-modern books
\end{enumerate}

\metroset{block=fill}
    \begin{block}{Goals for the next session}
    \begin{enumerate}
        \item understanding modelling theory relevant for digitization
        \item understanding the terms `digital edition', `digital archive' versus library catalogs
        \item[\textcolor{alert}{\faClose}] \texttt{<msDesc>} $\to$ next session
        \item[\textcolor{alert}{\faClose}] transcriptions in TEI \& Transkribus $\to$ later session
    \end{enumerate}
    \end{block}
\end{frame}


%-----------------------------------------------------






\section{Data modelling}


%------------------------------------------------------------------------------
\begin{frame}[allowframebreaks]{3 proporties of a model following Stachowiak}

\metroset{block=fill}
\begin{alertblock}{1)~ Mapping }
\begin{quote} \scriptsize
    „Modelle sind stets Modelle von etwas, nämlich Abbildungen, Repräsentationen natürlicher oder künstlicher Originale, die selbst wieder Modelle sein können.“ 
    %„Originale und Modelle werden hier ausschließlich als Attributklassen gedeutet, die oft die spezielle Gestalt attributiver Systeme erlangen.“ % 131
    „Der Abbildungsbegriff fällt mit dem Begriff der Zuordnung von Modell-Attributen zu Original-Attributen zusammen.“ \parencite[131--132]{stachowiak} % 132
    
    \alert{Models are always models of something, i.e. mappings from, representations of natural or artificial originals, that can be models themselves.}
\end{quote}
\end{alertblock}
\begin{alertblock}{2)~ Reduction}
\begin{quote} \scriptsize
    „Modelle erfassen im allgemeinen nicht alle Attribute des durch sie repräsentierten Originals, sondern nur solche, die den jeweiligen Modellerschaffern und/oder Modelbenutzern relevant scheinen.“ \parencite[132]{stachowiak}
    
    \alert{Models in general capture not all attributes of the original represented by them, but rather only those seeming relevant to their model creators and/ or model users.}
\end{quote}
\end{alertblock}
\framebreak 

\begin{alertblock}{3)~ Pragmatism}
\begin{quote} \scriptsize
    „Eine pragmatisch vollständige Bestimmung des Modellbegriffs hat nicht nur die Frage zu berücksichtigen, \emph{wovon} etwas Modell ist \lbrack{}Abbildungsmerkmal\rbrack{}, sondern auch, \emph{für wen, wann} und \emph{wozu} bezüglich seiner je spezifischen Funktionen es Modell ist.“ \parencite[132]{stachowiak} % 132
    
    \alert{ Models are not uniquely assigned to their originals per se. They fulfill their replacement function \\
    a) for particular – cognitive and/ or acting, model using subjects, \\
    b) within particular time intervals and \\
    c) restricted to particular mental or actual operations.}
\end{quote}
\end{alertblock}

\end{frame}

%------------------------------------------------------------------------------
\begin{frame}{Stachowiak's \emph{General Model Theory}}

\begin{block}{Stachowiak's notion of a model}
\begin{quote}
    Alle Erkenntnis ist Erkenntnis in Modellen oder durch Modelle und jede menschliche Weltbegegnung überhaupt bedarf des Mediums Modell.
\end{quote}
\alert{$\to$ All knowledge-making is knowledge-making in or through models and all human perception of the world needs models as a medium. }
\end{block}

\begin{block}{Data modelling}\small
Model = snippet of the real world but it only covers the attributes I chose to be relevant for the task at hand. 
Thus, the model and the aspect of the real world it models (its subject) diverge. 
\end{block}

\begin{block}{Digital, standardized \& formal models}\small 
Standardized models allow us to exchange and analyse data, search/query data. 
 $\to$ only \emph{formal models} can be processed digitally, i.e. every digital model is a formal model.
\end{block}

\alert{$\to$ Models are simplified representations of parts of the real world.}

\end{frame}

%------------------------------------------------------------------------------
\begin{frame}{Why should we care about modelling theory?}
    \begin{itemize}
        \item Modelling is a pivotal task in the Digital Humanities.
        \item When we create digital representations of material objects, those are models. 
        \item Models are by definition subjective, abstracted and not universal. 
        \item Their quality has to be judged in relation to their purpose (Stachowiak's 3rd criterion): $\to$
    \end{itemize}
\end{frame}


%------------------------------------------------------------------------------
    \begin{frame}{Different motivations for digital projects}
    \begin{block}{research-driven:}\small
        individualized for answering a research question, work-intensive \& relatively expensive.
    \end{block}

    \begin{block}{curation-driven}\small
        mass-digitization, cookie cutter approach which covers the most important elements for most use cases but can easily miss features relevant to subject-matter experts.
            \begin{itemize}\footnotesize
                \item despite the objects being digitized, users might have to go back to the material objects to fill in the blanks
                \item but objects become more discoverable \& (hopefully) somewhat comparable to larger corpora of similar objects
                \item superficial digitization can lead to the creation of misleading datasets, e.g. errors or bad tagging $\to$ \alert{reparative librarianship} is needed
                \item potential application for machine learning approaches
            \end{itemize}
    \end{block}

\end{frame}


%------------------------------------------------------------------------------
\begin{frame}{What are data?}
\metroset{block=fill}
\begin{block}{Definitions of `data'}
Plural of Latin \emph{datum} ('given'). 
\end{block}

\begin{itemize}\footnotesize 
    \item data aren't exactly a given but rather constructed or created.
    \item There has been a discussion in the DH whether we should call them \emph{capta}~\parencite{Drucker2011dataCapta}
    \item Most data (so-called `givens') are constructed by phenomenotechnical devices~\parencite{bachelard1968}, i.e. pereiving devices which translate the (often quite abtract) things they see into data.
    \item data has to be interpreted.
    \item Stachowiak's pragmatism criterion means we selectively capture data most important to us, not all possible aspects (!).
\end{itemize}

Data resulting from cataloging \& digitization contains interpretations \& is thus, 
\emph{per definitionem} always subjective \& incomplete (!).

\alert{data $\neq$ original!}

\end{frame}



% Teaser: Why are we here? General terms: digital edition and archive
\input{einheiten/digital-editing}\input{einheiten/digital-archives}

\section{Library catalog data}
\begin{frame}{Theuerdank (Graz Sondersammlungen -- Rara 1 III 11723 )}
\footnotesize
    Pfintzing, Melchior, et al. Die geuerlicheiten und einsteils der geschichten des loblichen streytparen und hochberümten helds und Ritters herr Tewrdannckhs. 1517.

    \includegraphics[width=0.95\textwidth]{img/theuerdank-biblio.png}
\end{frame}

%------------------------------------------------------------------------------
\begin{frame}[fragile,allowframebreaks]{MARC XML example}
    \footnotesize
    \href{https://lccn.loc.gov/2021667794/marcxml }{Library of Congress MARC XML for Theuerdank (simplified for demonstration purposes)}: \protect\url{https://www.loc.gov/item/2021667794}

    \begin{xmlcode}
<record xmlns="http://www.loc.gov/MARC21/slim" 
        xmlns:zs="http://docs.oasis-open.org/ns/search-ws/sruResponse">
    <leader>02892nam a22004093i 4500</leader>
    <controlfield tag="001">22061662</controlfield>

    <datafield ind1=" " ind2=" " tag="500">
      <subfield code="a">"BSB Shelfmark: Rar. 325 a"--Note extracted 
      from World Digital Library.</subfield>
    </datafield>
    
    <datafield ind1=" " ind2=" " tag="500">
      <subfield code="a">Original resource extent: 290 unnumbered 
      sheets : illustrations.</subfield>
    </datafield>
    [...]

</record>
\end{xmlcode}

    \begin{xmlcode}
<record xmlns="http://www.loc.gov/MARC21/slim" 
        xmlns:zs="http://docs.oasis-open.org/ns/search-ws/sruResponse">
        [...]
    
    <datafield ind1=" " ind2=" " tag="520">
      <subfield code="a">
      Among the many endeavors undertaken by the Holy Roman Emperor 
      Maximilian I (1459--1519) to further his legacy was his plan 
      of an epic retelling of his own life story in the form of 
      several works.       
      [...]
      Johann Schönsperger, a printer in Nuremberg, did the first, 
      very small print run in 1517, to be delivered to other princes 
      and sovereigns after the Emperor's death. 
      [...]
      Each of the 118 chapters is decorated by a xylograph 
      (wood engraving). The preparatory drawings for the xylographs 
      were created by the artists Leonhard Beck, 
      Hans Schäufelein, and Hans Burgkmair the Elder. The black-letter 
      type of the Theuerdank, designed by calligrapher Vinzenz Rockner, 
      was to become very influential for the development 
      of German typography.</subfield>
</datafield>

</record>
\end{xmlcode}

\end{frame}





% lead into XML and TEI


\section{Annotating with XML markup}


%-----------------------------------------------------
\begin{frame}{XML: eXtensible Markup Language}
\begin{columns}
\column{0.35\textwidth}
\begin{itemize}\small 
    \item \href{https://www.w3schools.com/xml/default.asp}{W3Schools Tutorial} 
    \item {paradigm of the separation of form and content} 
    \item {XML is a metalanguage}
\end{itemize}
\bgupper{w3schools}{black}{.xml} \\

\begin{itemize}\scriptsize 
    \item {RSS}, SOAP, XAML 
    \item {MathML}, {GraphML}~ 
    \item {XHTML}~
    \item {RDF}~
    \item {KML}~ 
    \item {Scalable Vector Graphics (SVG)}
\end{itemize}

\column{0.65\textwidth}
\metroset{block=fill}
\begin{block}{}
\begin{quote}
    \textbf{Extensible Markup Language (XML)} is a \textbf{markup language} and file format for storing, transmitting, and reconstructing arbitrary data. It defines a \textbf{set of rules for encoding documents} in a format that is \textbf{both human-readable and machine-readable.}  (\href{https://en.wikipedia.org/wiki/XML}{Wikipedia})
\end{quote}
\end{block}
\end{columns}

\end{frame}


%----------------------------------

\begin{frame}[fragile]{XML rules}
\begin{columns}
\column{0.42\textwidth}
\small
XML can be checked for \textbf{validity} (validation if it complies with a standard) and \textbf{well-formedness} (following the rules of XML) $\to$ will only be parsed if well-formed. Thus: \alert{Heed thy error messages!}\smallskip

There are rules on how elements can be named (you can look them up if relevant or will get informed by an error message). 

\bigskip

%\includegraphics[width=0.1\textwidth]{doppelkeks.jpg}~\bg{alert}{white}{Doppelkeks} ~ ~\bg{alert}{white}{russische Puppe}~\includegraphics[width=0.15\textwidth]{matroschka.jpg}\vspace{1em}

\mycommand{<key>value</key>}{XML as a key value notation} 
\column{0.55\textwidth}\footnotesize
\metroset{block=fill}
\begin{block}{Rules}
\begin{itemize}
    \item Hierarchical nesting below the root
    \item exactly one root element, i.e. one out-most russion doll
    \item start and end tag 
    \item tag names are case-sensitive (!) 
    \item empty elements allowed (\& can be shortened) 
\end{itemize}
\end{block}
\bigskip 

\begin{block}{Minimal example}
\begin{xmlcode}
<?xml version="1.0" ?>
<root>
  <element attribute="value">
    content
  </element>
  <!-- comment -->
</root>
\end{xmlcode}
\end{block}
\end{columns}

\end{frame}


%----------------------------------


\begin{frame}[fragile,allowframebreaks]{XML rules}

\small

\bg{w3schools}{white}{Prolog}~ \\
\mycommand{<xml version="1.0" encoding="utf-8">}{XML declaration}
\mycommand{<?xsl-stylesheet type="text/xsl" href="my.xsl"?>}{processing instructions  (optional)}
\bigskip

you can include document models (optional) \\
DTD, XML Schema, RelaxNG, Schematron 
\bigskip

\bg{w3schools}{white}{entities}~ `protected' characters that have a meta meaning in XML like: \\
\mycommand{&lt;}{<}
\mycommand{&gt;}{>}
\mycommand{&amp;}{\&}


\end{frame}


%-----------------
\begin{frame}{XML family and vocabularies}
\begin{columns}
\column{0.45\textwidth}
\footnotesize
\bg{w3schools}{white}{XML}~structured description of data \\
\bg{w3schools}{white}{XPath}~navigating xml documents \\
\bg{w3schools}{white}{XML Schema}~strict data model \\
\bg{w3schools}{white}{XSL}~Extensible Stylesheet Language  \\
\bg{w3schools}{white}{XSLT}~XSL-Transformations, i.e. transforming XML documents  \\
\bg{w3schools}{white}{XSL-FO}~ formatted output (e.g. print) \\
\bg{w3schools}{white}{XQuery}~query language for XML databases \\
\bg{w3schools}{white}{and more}~
\column{0.45\textwidth}
\metroset{block=fill}
\begin{block}{}
\footnotesize
\begin{itemize}
    \item \textbf{(X)HTML} Hypertext Markup Language 
    \item \textbf{EAD} Encoded Archival Description 
    \item \textbf{TEI} Text Encoding Initiative 
    \item \textbf{CEI} Charters Encoding Initiative 
    \item \textbf{MEI} Music Encoding Initiative 
    \item \textbf{LIDO} Lightweight Information Describing Objects (describing museum or collection objects)
    \item \textbf{SVG} Scalable Vector Graphics 
    \item \textbf{KML} Keyhole Markup Language (geography)
    \item \textbf{MathML} 
    \item \textbf{CML} Chemical Markup Language, \dots
\end{itemize}
\end{block}
\end{columns}


\end{frame}


%------------------------------------------------------------------------------

\begin{frame}[standout]
  \alert{Practice!} \\
  \normalsize
  Open a new XML document in your editor (Oxygen). \\
  Create new elements and find 3 ways to break it so that you get an error. \\
  Then fix the error.

\end{frame}

\input{einheiten/tei-publication-solutions}
\input{einheiten/xml-tei-intro}

\section{Transcriptions using Transkribus}



%-----------------------------------------------------
\begin{frame}[allowframebreaks,fragile]{Transcription}
\subsubsection{Transcription}
\metroset{block=fill}

\begin{columns}
\column{0.48\textwidth}
\begin{itemize}\small
\item  OCR (Optical Character Recognition) -- e.g. Transkribus (transcription support)
\item  Transkribus Keyword Spotting 
\item  fuzzy search which should also find the word if it's mistranscribed
\item  Writer identification
\end{itemize}

\column{0.48\textwidth}

\includegraphics[width=\textwidth]{img/ocr1.png}
\includegraphics[width=\textwidth]{img/ocr2.png}
\end{columns}

\framebreak 

\begin{columns}
\column{0.48\textwidth}
\begin{block}{Typical phenomena}
\begin{itemize}
\item “Special characters"
\item Abbreviations
\item damaged or unreadable text
\item additions, deletions, substitutions, corrections
\item editorial interventions (emendations and conjectures)
\item  editorial additions or omissions
\end{itemize}
\end{block}

\column{0.48\textwidth}

\includegraphics[width=\textwidth]{img/ocr3.png}
\includegraphics[width=\textwidth]{img/ocr4-transkribus-app.png}

\end{columns}

\end{frame}
%-----------------------------------------------------

\begin{frame}{What is Transkribus?}
    \begin{block}{Transkribus}
       \begin{quote}
           \dots{}is a comprehensive platform for the digitisation, AI-powered text recognition, transcription and searching of historical documents. (\href{https://readcoop.eu/transkribus/}{source})
       \end{quote}
    \end{block}
    \begin{itemize}
        \item we're using the web version TranskribusLite: \protect\url{https://transkribus.eu/lite}
        \begin{itemize}
            \item for more complexity (which you might not need), download the software (Transkribus eXpert)
            \item Lite is easier to learn \& has only the essentials.
            \item you need an account and to buy credits after you have used up your initial 200
            \item print \& manuscript text recognition have different pricing
            \item there are stipends
        \end{itemize}        
    \end{itemize}
\end{frame}


%-----------------------------------------------------
\begin{frame}{Transkribus: Lite versus eXpert}
    \includegraphics[width=0.95\textwidth]{img/transkribus-versions.png}
\end{frame}
%-----------------------------------------------------
\begin{frame}{Transkribus: How to guides}
    \includegraphics[width=0.95\textwidth]{img/transkribus-resources.png}
\end{frame}
%-----------------------------------------------------
\begin{frame}{Transkribus: Creating transcriptions to train your own model}
    \includegraphics[width=0.95\textwidth]{img/transkribus-training1.png}
    \begin{itemize}\small
        \item first, transcribe a number of pages 
        \begin{itemize}\footnotesize
            \item recommended: 25--75
            \item depending on print or handwritten
            \item you can build on base models (should be similar)
            \item you can speed up the process by creating a model, running it on new pages, correcting them and then repeating the process
            \item correcting is usually still faster than transcribing from scratch
        \end{itemize}
        \item be mindful to adhere to the transcription guidelines you want the model to learn
        \item train your model using the gold-standard transcriptions
        \item use the model on the pages you want transcribed
        \item there will probably be errors: fix them \& use the extra training data created thus to improve the model by retraining it
        \item consider publishing your model if it's of reusable quality
    \end{itemize}
\end{frame}
%-----------------------------------------------------
\begin{frame}{Transkribus: Configuring your model}
    \includegraphics[width=0.95\textwidth]{img/transkribus-training2.png}
\end{frame}


%-----------------------------------------------------
\begin{frame}{Transkribus: further resources}
    \begin{itemize}
        \item \emph{How to historical text recognition: A Transkribus Quickstart Guide}, \LaTeX{}-Ninja Blog, 10. November 2019, \href{https://latex-ninja.com/2019/11/10/how-to-historical-text-recognition-a-transkribus-quickstart-guide/}{URL}. \alert{$\to$ how to reuse exisiting models for print on the example of the Noscemus GM4}
        \item \emph{Training my own Handwritten Text Recognition (HTR) model on Transkribus Lite}, \LaTeX{}-Ninja Blog, 22. March 2022, \href{https://latex-ninja.com/2022/03/22/training-my-own-handwritten-text-recognition-htr-model-on-transkribus-lite/}{URL}. \alert{$\to$ experiences training my own model}
    \end{itemize}
\end{frame}


\input{einheiten/tei-for-digital-editing}

%

%-----------------------------------------------------
\begin{frame}[allowframebreaks, fragile]{Named Entities}
\subsubsection{Named Entities, normalization and norm data}

\begin{block}{Persons vs. personal names}\small 
A \texttt{<person>} (person themselves) isn't identical with their \texttt{<persName>} (name = word = string of characters referring to a person)! 
\medskip

There are the following elements in the TEI: \texttt{<persName>} for \texttt{<person>}, \texttt{<orgName>} for \texttt{<org>} (organisation), \texttt{<placeName>} for \texttt{<place>}.\\

\texttt{<geogName>} (= geographical) = landscape markers (such as mountains, etc.) 
\end{block}

\framebreak

\begin{block}{Normalizing different names forms}\small
Problem: \textbf{Different name forms}, thus:
\bg{w3schools}{white}{Normalization:} Showing that \emph{one and the same} person is meant by giving out \textbf{identification (ID) numbers} on the internet and normalizing name forms. We can add that as a reference (\texttt{@ref}) to a list of persons in the TEI header, for example. 

We reference the information in this list using its \texttt{@xml:id} in \texttt{@ref} attribute in-text. \\

We can add norm data (such as GND, \emph{Gemeinsame Normdatei}, or \href{https://www.geonames.org/}{geonames}) using attributes, e.g. \texttt{@n} (\emph{label}) or \texttt{@ana} (interpretation). 
\end{block}

\green{\href{https://explore.gnd.network/search?term=michael\%20maier\&rows=25}{Try the GND explorer!}}

\framebreak

\bg{w3schools}{white}{Redundancy is a source of errors!} $\to$ reference one place where you can easily check if information is correct or update in just one place in case of changes -- store the information exactly once, if possible. 

$\to$ Mistakes happen and this way, you will only need to fix them once in one place, not 200 occurences. 
For unique references, use  \\
\verb|<person xml:id="Mina">Mina</person>|.

\begin{block}{\texttt{xml:id}}\footnotesize
  You can only have the same value for the \texttt{@xml:id} once in your document!  \\
You reference it using the \texttt{@ref} attribute prefaced by a hashtag (shorthand for `in this document' in the TEI): \\
\verb|<persName ref="#Mina">Mina</persName>|.
\end{block}

\end{frame}

%------------------------------------------------------------------------------
\begin{frame}[allowframebreaks,fragile]{\texttt{<tei:name>}}
The TEI offers a whole set of elements to denote names such as:


  \begin{columns}[T,onlytextwidth]
    \column{0.5\textwidth}
\begin{itemize}
    \item persName
    \item surname
    \item firstname
    \item name
\end{itemize}

\column{0.5\textwidth}
You could also simply use \texttt{<name>} and \texttt{@type} attribute to define which type of name it is. 
\begin{xmlcode}
<name type=„person“>
<name type=„pet“>
<name type=„vulgo“>
<name type=„house“>
<rs type=„name“>
\end{xmlcode}
\end{columns}

\texttt{<tei:name>} = 
\begin{itemize}
    \item concrete version of \texttt{<tei:rs>} (refering string)
    \item generalisation of \texttt{<persName>} (personal name)
\end{itemize}

\framebreak

  \begin{columns}[T,onlytextwidth]
    \column{0.5\textwidth}
      \begin{itemize}
    \item surname 
    \item forename 
    \item roleName (social role, e.g. King of France)
    \item addName (additional name) 
    \item nameLink (name link)
    \item genName (generational name component, e.g. sen., jun.)
    \item orgName (organization name)
    \item placeName
    \item geogName (geographical name)
\end{itemize}

    \column{0.5\textwidth}

     ``Eberhard, count of Württemberg''
\begin{xmlcode}
<persName>
  <foreName>Eberhard</foreName>
  <roleName>count of
    <placeName>Württemberg</placeName>
  </roleName>
</persName>
\end{xmlcode}

``Count Eberhard of Württemberg and his sons Ludwig and Ulric'' etc.


  \end{columns}
\end{frame}




%------------------------------------------------------------------------------
\begin{frame}[fragile]{Unique identifiers, noramlization, norm data\dots}

  \begin{columns}[T,onlytextwidth]
    \column{0.5\textwidth}
      \green{tei:placeName @key} points to the register of persons:
\begin{xmlcode}
<persName key="A000835">
  Heuschkel</persName>
\end{xmlcode}

which is the same as: \\
\protect\url{http://www.weber-gesamtausgabe.de/de/A000835}

or:
\begin{xmlcode}
<person xml:id="A000835">
\end{xmlcode}

    \column{0.5\textwidth}

      \metroset{block=fill}

      \begin{block}{modernisation?}\footnotesize
        Vienna = Wien, capital of Austria
      \end{block}

      \begin{alertblock}{main entry point?}\footnotesize
        Preßburg $\to$ Bratislava, capitol of Slovakia \\
also: Preßburg, Pozsony, Prešporok
      \end{alertblock}

      \begin{exampleblock}{identifier?}\scriptsize
        Bernardus Papiensis $\to$ VIAF-ID: 15126540 $\to$\protect\url{<http://viaf.org/viaf/15126540/>}

\textbf{main entry point:} Bernhard von Pavia, before 1150-18.9.1213
also: Bernardus Papiensis, Bernardo Balbi, Bernardus Balbus, Bernard of Pavie, Bernardus Circa, Bernardus praepositus Faventinus, Bischof Bernhard von Faenza, Bischof Bernhard von Pavia \dots
      \end{exampleblock}

  \end{columns}
\end{frame}


%------------------------------------------------------------------------------
\begin{frame}[allowframebreaks]{Norm data}
  The Virtual International Authority File (VIAF) is an international authority file.

  \begin{columns}[T,onlytextwidth]
    \column{0.5\textwidth}
      \begin{block}{Identifiers}
        Describe historical individuals (or places etc.) with a unique identifier using authority files or other norm data. 
Datasets linking information in that way are called \emph{Linked Open Data}, for example: Wikipedia pages list authority control information, collect information as Linked Open Data (Wikidata)
      \end{block}

    \column{0.5\textwidth}

      \metroset{block=fill}

      \begin{block}{Example 1}
        \includegraphics[width=0.95\textwidth]{img/mmaier-wiki1.png}
      \end{block}
      
            \begin{block}{Example 2}
        \includegraphics[width=0.95\textwidth]{img/mmaier-wiki2.png}
      \end{block}


  \end{columns}
  \framebreak
  
    \begin{columns}[T,onlytextwidth]
    \column{0.5\textwidth}
        \begin{alertblock}{Authority control}
        \begin{quote}
    In library science, \textbf{authority control} is a process that organizes bibliographic information, for example in library catalogs by using a single, distinct spelling of a name (heading) or a numeric identifier for each topic. (\href{https://en.wikipedia.org/wiki/Authority_control}{Wikipedia})
\end{quote}
      \end{alertblock}

      \begin{exampleblock}{Examples for norm data}\footnotesize
\begin{itemize}
    \item German \textbf{GND} (Gemeinsame Normdatei)
    \item \textbf{LCCN} (Library of Congress Control Number)
    \item \textbf{VIAF} (Virtual International Authority File)
\end{itemize}
      \end{exampleblock}
      
       \column{0.5\textwidth}
             
            \begin{block}{Example 3}
        \includegraphics[width=0.95\textwidth]{img/mmaier-wiki3.png}
      \end{block}

      \end{columns}
\end{frame}




%-----------------------------------------------------
\begin{frame}[allowframebreaks]{Reminder: Making the TEI your own}
\small
\metroset{block=fill}

\begin{columns}
\column{0.5\textwidth}
    \begin{block}{How to find information on TEI elements}
    \dots and teach yourself how to use new elements:
\begin{itemize}\footnotesize
    \item General TEI guidelines (\href{https://tei-c.org/release/doc/tei-p5-doc/en/html/SG.html}{XML Primer}, \href{https://tei-c.org/support/learn/}{Learn the TEI page}, etc.)
    \item web-search TEI + (element you want to know about), i.e. ``tei teiHeader'' and you will get:
    \begin{enumerate}\scriptsize
        \item \href{https://www.tei-c.org/release/doc/tei-p5-doc/en/html/ref-teiHeader.html}{definition page}
        \item \href{https://www.tei-c.org/release/doc/tei-p5-doc/en/html/examples-teiHeader.html}{list of all examples for that element} $\to$ directly over websearch or click `show all' in the examples on the `definitons page'
        \item sometimes even an \href{https://www.tei-c.org/release/doc/tei-p5-doc/en/html/HD.html}{module overview text for things as big as \texttt{<teiHeader>}} (has its own module)
    \end{enumerate}
\end{itemize}
    \end{block}


\column{0.45\textwidth}

\begin{block}{Module 13:}
\href{https://tei-c.org/release/doc/tei-p5-doc/en/html/ND.html}{Names, Dates, People, and Places}.
\end{block}

\begin{block}{}
\footnotesize
Also: The TEI guidelines are documentation and reference, not necessarily ideal teaching tools $\to$ overwhelming. 
Maybe try other tutorials like the \href{http://gams.uni-graz.at/o:dhoxss2016-tei-names}{these slides}.
\end{block}

\end{columns}

\end{frame}



\begin{frame}[fragile]{Keyboard Shortcuts}
    
\begin{itemize}
    \item Repetitive tasks are computer tasks
    \item \textbf{Oxygen:} CTRL-F $\to$ select „regular expression“
    \item Mark up all words
    \begin{enumerate}
        \item Search for: \verb|\b(\w+)\b|
        \item Replace with \verb|<w>$1</w>|
    \end{enumerate}
    \item Mark up all lines
    \begin{enumerate}
        \item Search for \verb|\s{20}([\S].*?)\n|
        \item Replace with \verb|<l>$1</l>\n|
    \end{enumerate}
\end{itemize}
\end{frame}

%----------------------------------------------

\begin{frame}{Regular expressions (RegEx)}
 \begin{multibox}{2} % Anzahl der Boxen in einer Reihe angeben
\begin{subbox}{subbox}{keyboard shortcuts}\footnotesize

\mycommand{CTRL+C}{copy}
\mycommand{CTRL+Z}{undo}
\mycommand{CTRL+Y}{redo}
\mycommand{CTRL+A}{all}
\mycommand{CTRL+S}{save}

\mycommand{Alt+Tab}{jump between windows}
\mycommand{CTRL+Tab}{jump between tabs}
\mycommand{F5}{Browser-Refresh}

\end{subbox}
\begin{subbox}{customcolor}{keyboard shortcuts}\footnotesize

\mycommand{CTRL+arrow}{jump-navigate}
\mycommand{CTRL+SHIFT+arrow}{jump-markup)}
\mycommand{CTRL+SHIFT+e}{new element in (<oXygen/>)}

\end{subbox}
\end{multibox}
\end{frame}


%---------------------------------------------


\begin{frame}[fragile,allowframebreaks]{RegEx}
\footnotesize


\begin{multibox}{2} % Anzahl der Boxen in einer Reihe angeben
\begin{subbox}{subbox}{put afterwards}\footnotesize

\mycommand{nothing}{exactly 1x}
\mycommand{?}{1x or zero}
\mycommand{*}{as often as you want}
\mycommand{+}{at least 1x}
\mycommand{{n}}{n-times}

\end{subbox}
\begin{subbox}{customcolor}{selection}\footnotesize

\mycommand{.}{any character}

\mycommand{[abc]}{choice of chars}
\mycommand{[a-z][0-9]}{digital and char choice}
\mycommand{\n}{newline}

\end{subbox}
\end{multibox}

%---------------------------------------------
\begin{multibox}{2} % Anzahl der Boxen in einer Reihe angeben
\begin{subbox}{subbox}{Grouping}\footnotesize

\mycommand{()}{to group, to access: (\$1)}
\mycommand{(.*?)}{anything (non-greedy)}


\mycommand{^}{negates the following}
\mycommand{^}{start of string}
\mycommand{\$}{end of string}
\mycommand{\\}{escape sequence}
\mycommand{|}{or pipe}
\end{subbox}
\begin{subbox}{subbox}{RegEx}\footnotesize
\mycommand{\s}{\emph{space}: space, tab, newline}
\mycommand{\d}{ (\emph{digit}}
\mycommand{\w}{word (=letter,nr,underscore}
\mycommand{\D \W \S}{the opposite as in lowercase}

\end{subbox}
\end{multibox}

\end{frame}




% Practical annotation
%\input{einheiten/annotation}\input{einheiten/datatypes-metadata-standards}
%

\begin{frame}[fragile]{Keyboard Shortcuts}
    
\begin{itemize}
    \item Repetitive tasks are computer tasks
    \item \textbf{Oxygen:} CTRL-F $\to$ select „regular expression“
    \item Mark up all words
    \begin{enumerate}
        \item Search for: \verb|\b(\w+)\b|
        \item Replace with \verb|<w>$1</w>|
    \end{enumerate}
    \item Mark up all lines
    \begin{enumerate}
        \item Search for \verb|\s{20}([\S].*?)\n|
        \item Replace with \verb|<l>$1</l>\n|
    \end{enumerate}
\end{itemize}
\end{frame}

%----------------------------------------------

\begin{frame}{Regular expressions (RegEx)}
 \begin{multibox}{2} % Anzahl der Boxen in einer Reihe angeben
\begin{subbox}{subbox}{keyboard shortcuts}\footnotesize

\mycommand{CTRL+C}{copy}
\mycommand{CTRL+Z}{undo}
\mycommand{CTRL+Y}{redo}
\mycommand{CTRL+A}{all}
\mycommand{CTRL+S}{save}

\mycommand{Alt+Tab}{jump between windows}
\mycommand{CTRL+Tab}{jump between tabs}
\mycommand{F5}{Browser-Refresh}

\end{subbox}
\begin{subbox}{customcolor}{keyboard shortcuts}\footnotesize

\mycommand{CTRL+arrow}{jump-navigate}
\mycommand{CTRL+SHIFT+arrow}{jump-markup)}
\mycommand{CTRL+SHIFT+e}{new element in (<oXygen/>)}

\end{subbox}
\end{multibox}
\end{frame}


%---------------------------------------------


\begin{frame}[fragile,allowframebreaks]{RegEx}
\footnotesize


\begin{multibox}{2} % Anzahl der Boxen in einer Reihe angeben
\begin{subbox}{subbox}{put afterwards}\footnotesize

\mycommand{nothing}{exactly 1x}
\mycommand{?}{1x or zero}
\mycommand{*}{as often as you want}
\mycommand{+}{at least 1x}
\mycommand{{n}}{n-times}

\end{subbox}
\begin{subbox}{customcolor}{selection}\footnotesize

\mycommand{.}{any character}

\mycommand{[abc]}{choice of chars}
\mycommand{[a-z][0-9]}{digital and char choice}
\mycommand{\n}{newline}

\end{subbox}
\end{multibox}

%---------------------------------------------
\begin{multibox}{2} % Anzahl der Boxen in einer Reihe angeben
\begin{subbox}{subbox}{Grouping}\footnotesize

\mycommand{()}{to group, to access: (\$1)}
\mycommand{(.*?)}{anything (non-greedy)}


\mycommand{^}{negates the following}
\mycommand{^}{start of string}
\mycommand{\$}{end of string}
\mycommand{\\}{escape sequence}
\mycommand{|}{or pipe}
\end{subbox}
\begin{subbox}{subbox}{RegEx}\footnotesize
\mycommand{\s}{\emph{space}: space, tab, newline}
\mycommand{\d}{ (\emph{digit}}
\mycommand{\w}{word (=letter,nr,underscore}
\mycommand{\D \W \S}{the opposite as in lowercase}

\end{subbox}
\end{multibox}

\end{frame}





% Relevant technologies introduction
%\input{einheiten/html-bootstrap}
%\input{einheiten/latex-reledmac}

% get to XPath and XSLT
%\input{einheiten/xpath}
%\input{einheiten/xslt}


%TODO in the future: Advanced XSLT for-each group, param, call template, variables
%TODO maybe: Primar Schemata
%TODO see leftover-text.tex

%------------------------------------------------------------------------------
% BACKMATTER
%------------------------------------------------------------------------------
\frame[allowframebreaks]{\AtNextBibliography{\footnotesize}\printbibliography}

\end{document}
